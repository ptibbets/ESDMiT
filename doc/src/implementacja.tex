\section{Implementacja w~\textit{C++}}

\subsection{Wstęp do prac}
\indent

W~związku z~decyzją podjętą na łamach całej grupy, do implementacji wybrana
została biblioteka \textit{Eigen}. Wykorzystana została najnowsza (na dzień
tworzenia projektu) wersja oznaczona jako 3.3.1. Dodatkowo, w~celu wykorzystania
przykładowych rzeczywistych sygnałów pochodzących ze strony \cite{EKG-DB}
bezpośrednio w~kodzie \textit{C++}, niezbędne było wykorzystanie biblioteki
\textit{WFDB} w~wersji 10.5.24, która obsługuje format plików udostępnionych na
tej stronie. Punktem wyjściowym do implementacji był prototyp algorytmu
stworzony w~programie \textsc{Matlab}, który przedstawiony został na
listingu~\ref{lst:emdm}. W~związku z~brakiem odpowiedników pewnych funkcji
programu \textsc{Matlab} w~bibliotece \textit{Eigen}, niezbędne było ich
zaimplementowanie.

\subsection{Przeniesienie kodu do języka \textit{C++}}
\indent

Pierwszym etapem prac było skompilowanie i~przetestowanie biblioteki
\textit{WFDB}. Następnie, po sprawdzeniu poprawności działania, stworzona
została klasa \textit{DataReader}, która wczytuje wybrany plik pochodzący ze
strony~\cite{EKG-DB} i~zwraca zawarte w~nim dane jako kontener biblioteki
\textit{Eigen} --~\textit{VectorXd}.

W~kolejnym kroku, dzięki posiadaniu danych w~postaci porównywalnej jak
w~środkowisku \textsc{Matlab}, możliwe było odwzorowywanie kodu
z~listingu~\ref{lst:emdm}. Niezbędne okazało się stworzenie funkcji, która
realizowałaby operację ilorazu różnicowego na danych znajdujących się
w~wektorze. Analogicznie jak w~programie \textsc{Matlab}, nazwana ona została
\textit{diff}. Przedstawiona ona została na listingu~\ref{lst:diff}. Inną
brakującą funkcjonalnością było poszukiwanie lokalnych maksimów i~minimów
funkcji, której kolejne wartości zawarte byłyby w~wektorze --~funkcja
\textit{findpeaks} w~programie \textsc{Matlab}.

Bibilioteka \textit{Eigen} posiada realizację funkcji sklejanych, lecz w~związku
z~jej nadmiernym rozbudowaniem w~stosunku do wymagań projektu, została ona
wykorzystana do stworzenia klasy \textit{Spline}, która udostępnia jedynie
wymaganą funkcjonalność --~listing~\ref{lst:spline}. Funkcje sklejane
wykorzystywane były do interpolowania obwiedni sygnału na podstawie lokalnych
ekstremów, co zostało wyodrębnione do oddzielnej funkcji \textit{getEnv}.
Odchylenie standardowe $SD$ liczone było w~oddzielnej funkcji \textit{getNewSD}.

Po wykonaniu powyższych operacji, główna część algorytmu EMD odpowiedzialna za
wyszukiwanie IMF mogła zostać sprowadzona do implementacji składającej się
z~kilku podstawowych operacji na wektorach oraz wywołań opisanych wcześniej
funkcji co poprawiło przejrzystość kodu, umożliwiło przetestowanie
poszczególnych jego fragmentów oraz ułatwiło rozwiązywanie błędów.

Efekty powyższych prac umieszczone są w repozytorium \textit{GitHub}~\cite{GIT}.
W~niniejszym rozdziale przedstawione zostały wyłącznie wybrane fragmenty
implementacji o~kluczowym znaczeniu dla działania programu. Kompletny kod
źródłowy znajduje się w~powyższym repozytorium.